%%%%%%%%%%%%%%%%%%%%%%%%%%%%%%%%%%%%%%%%%
% Industry CV for AI Research Scientist
%%%%%%%%%%%%%%%%%%%%%%%%%%%%%%%%%%%%%%%%%

\documentclass{resume}

\usepackage[left=0.75in,top=0.6in,right=0.75in,bottom=0.6in]{geometry}
\usepackage[colorlinks,urlcolor=blue]{hyperref}
\usepackage[style=numeric-comp,maxbibnames=99,sorting=ydnt]{biblatex}
\usepackage{comment}
\addbibresource{paper.bib}

\name{Yidi Qi}
\address{y.qi@northeastern.edu}
\address{\url{https://yidiq7.github.io}}
%\address{\url{https://yidiq7.github.io} \quad\textbullet\quad \url{https://github.com/yidiq7}}
\address{360 Huntington Ave \\ Boston, MA 02215}

\begin{document}

%----------------------------------------------------------------------------------------
%   EDUCATION
%----------------------------------------------------------------------------------------
\begin{rSection}{Education}
{\bf Northeastern University} \hfill {Boston, MA} \\
Ph.D. in Physics \hfill {\em Sep. 2021 - Present (Expected 2026)} \\
{\em Research Focus: AI for String Theory Mathematics}

{\bf Stony Brook University} \hfill {Stony Brook, NY} \\
M.A in Physics \hfill {\em Sep. 2018 - Sep. 2021} \\

{\bf Jilin University} \hfill {Changchun, China} \\
B.S. in Physics\hfill {\em Sep. 2014 - June 2018} \\
{\em Tang Aoqing Honors Program in Science}
\end{rSection}

%----------------------------------------------------------------------------------------
%   RESEARCH HIGHLIGHTS
%----------------------------------------------------------------------------------------
\begin{rSection}{Selected AI Research Projects}

\begin{rSubsection}{From High-Dimensional PDE Solving to Formal Verification}{}{}{}{}
\item Designed Physics-Informed Neural Networks (PINNs) to solve challenging systems of nonlinear PDEs on high-dimensional manifolds (Calabi-Yau and G2 metrics), significantly improving computational efficiency over traditional methods.
\item Created and maintain \href{https://github.com/yidiq7/MLGeometry}{MLGeometry}, a Python package for ML-based differential geometry.
\item Developing a validation pipeline to formally verify neural network solutions, transforming high-precision numerical outputs into rigorous, machine-verifiable proofs of existence.
\end{rSubsection}

\begin{rSubsection}{Interpreting "Black Box" Models for Scientific Discovery}{}{}{}{}
\item Probed the internal mechanism of a CNN trained to solve a central problem in number theory: predicting the rank of elliptic curves from their coefficient sequences.
\item Through a series of custom, theory-driven experiments, demonstrated that the model develops a sophisticated hybrid methodology, combining established mathematical heuristics with subtle, long-range patterns ("murmurations") in the data.
\end{rSubsection}

\begin{rSubsection}{Discovering Novel Geometric Structures with Quality-Diversity Algorithms}{}{}{}{}
\item Applied Quality-Diversity (QD) optimization, a modern class of evolutionary algorithms to discover novel Special Lagrangian submanifolds, a key target structure in differential geometry and string theory.

\end{rSubsection}
\begin{rSubsection}{Reinforcement Learning for Combinatorial Search on Graphs}{}{}{}{}
\item Architected a deep reinforcement learning framework, inspired by AlphaZero, to find optimal paths for a computationally hard graph transformation problem previously intractable with classical algorithms.
\item Designed a novel two-network system where an RL-based policy network guides the search, and a Graph Neural Network (GNN) acts as a learned heuristic, estimating distance-to-target.
\end{rSubsection}

\end{rSection}

%----------------------------------------------------------------------------------------
%   PROFESSIONAL EXPERIENCE
%----------------------------------------------------------------------------------------
%\pagebreak
\begin{rSection}{Experience}
\begin{rSubsection}{The NSF AI Institute for AI and Fundamental Interactions (IAIFI)}{Cambridge, MA}{Junior Investigator}{Jan. 2023 - Present}{}
\item Apply cutting-edge AI techniques to solve fundamental problems in math and theoretical physics.
\item Design, implement, and test novel machine learning models for tasks including PDE solving, combinatorial optimization, and scientific model interpretation.
\end{rSubsection}

\begin{rSubsection}{Simons Center for Geometry and Physics}{Stony Brook, NY}{Research Assistant}{May 2019 - Sep. 2021}{}
\item Conducted foundational research on applying neural networks to problems in complex geometry.
%\item Developed a custom neural network architecture to approximate solutions to complex geometric equations, significantly outperforming traditional numerical methods in efficiency.
\end{rSubsection}

\begin{rSubsection}{ATLAS Experiment at Stony Brook University and CERN}{Geneva, Switzerland}{Research Assistant}{Sep. 2016 - May 2019}{}
\item Performed statistical analysis and developed machine learning classifiers (BDT, NN) for signal-background separation on Large Hadron Collider data.

%\item Engineered ML classifiers (Boosted Decision Trees, NNs) for signal-background separation in the Higgs to di-muon decay channel.
%\item Contributed to a high-impact ATLAS collaboration paper in \href{https://journals.aps.org/prl/abstract/10.1103/PhysRevLett.119.051802}{\textit{Physical Review Letters}}.
\end{rSubsection}
\end{rSection}



%------------------------------------------------

\begin{rSection}{Research Visits}
\begin{rSubsection}{University of Cambridge, Department of Computer Science}{Cambridge, UK}{Visitor}{Sep. 2023 - Oct. 2023}{Host: Challenger Mishra}
\end{rSubsection}

\begin{rSubsection}{King's College London, Department of Mathematics}{London, UK}{Visitor}{Aug. 2023 - Sep. 2023}{Host: Daniel Platt}
\end{rSubsection}   
\end{rSection}

\begin{pSection}{Publications}
\setlength{\bibitemsep}{4pt}
\nocite{*}
\textbf{Authors are listed in alphabetical order}
%\textit{Authors are listed in alphabetical order}
\printbibliography[heading=none] 
\end{pSection}



\begin{rSection}{Talks}
\talk{String Phenomenology 2025}{Northeastern University}{Searching for New Special Lagrangians with Quality-Diversity Optimization}{July 2025}

\talk{String Data 2024}{YITP, Kyoto University}{Harmonic 1-form on real loci of Calabi-Yau manifolds}{Dec. 2024}

\talk{Forum for Young Scholars in Physics}{Jilin University}{An introduction to String theory and Artificial Intelligence}{Dec. 2024}

\talk{A Day of Deep Learning and High Energy Theory}{Northeastern University}{Numerical Calabi-Yau and G2 Metrics from Neural Networks}{Mar. 2024}

\talk{AI/Physics Journal Club}{Queen Mary University of London}{Solving PDEs on Higher Dimensional Manifolds with Neural Networks}{Nov. 2023}

\talk{ML@CL Ad-hoc Seminar Series}{University of Cambridge}{Solving PDEs on Higher Dimensional Manifolds with Neural Networks}{Oct. 2023}


\talk{Boston Area Chinese Young Physicists Seminar}{Harvard University}{Machine Learning and String Theory for Babies}{Oct. 2022}

\talk{Workshop on Machine Learning and Mathematical Conjecture}{CMSA, Harvard University}{Tutorial on Machine Learning and Knot Theory}{Apr. 2022}

\talk{Seminar Seires on String Phenomenology}{Online}{Numerical Calabi-Yau Metrics from Holomorphic Networks}{Feb. 2021}

%\talk{Graduate Seminar}{Stony Brook University}{Representations  of  Quantum  Many-body  Systems  with  Neural  Networks}{Apr. 2020}

%\talk{Graduate Seminar}{Stony Brook University}{Dark Matter at the Large Hardon Collider}{Apr. 2019}

%\item Select ggF events with MVA methods in Higgs to Di-muon Analysis, Stony Brook University and CERN {\em Aug. 2017}
\end{rSection}

\iffalse
\begin{rSection}{Conference Attendance}


\conf{IAIFI Summer School \& Workshop}{Tufts University, Aug. 2022}

\conf{Computational Differential Geometry and its Applications in Physics}{Simons Center for Geometry and Physics, Stony Brook University, Nov. 2022}

%\item Select ggF events with MVA methods in Higgs to Di-muon Analysis, Stony Brook University and CERN {\em Aug. 2017}
\end{rSection}
\fi

%----------------------------------------------------------------------------------------
%   TECHNICAL SKILLS
%----------------------------------------------------------------------------------------
\begin{rSection}{Technical Skills}
\begin{tabular}{ @{} >{\bfseries}l @{\hspace{6ex}} l }
AI Specialties & Deep Learning (CNNs, GNNs), Reinforcement Learning, PINNs, Interpretability \\
& Evolutionary Algorithms (Quality-Diversity Optimization) \\
Programming & Python, C/C++, Fortran \\
Frameworks & TensorFlow, PyTorch, JAX \\
Tools & Linux (Bash), Git, Slurm, \LaTeX, Mathematica \\
\end{tabular}
\end{rSection}


\begin{rSection}{Mentoring Experience}
\talk{LOGML Summer School 2024}{Imperial College London}{Calabi-Yau Metrics with $U(1)$-invariant Neural Networks}{July 2024}

\talk{Machine Learning in Mathematics \& Theoretical Physics 2023}{University of Oxford}{Tutorial on Calabi-Yau Manifolds and Ricci-flat Metrics}{July 2023}

\end{rSection}
%----------------------------------------------------------------------------------------
%	EXAMPLE SECTION
%----------------------------------------------------------------------------------------

%\begin{rSection}{Section Name}

%Section content\ldots

%\end{rSection}

%----------------------------------------------------------------------------------------

\end{document}
